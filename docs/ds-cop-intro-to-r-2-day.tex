\PassOptionsToPackage{unicode=true}{hyperref} % options for packages loaded elsewhere
\PassOptionsToPackage{hyphens}{url}
%
\documentclass[]{book}
\usepackage{lmodern}
\usepackage{amssymb,amsmath}
\usepackage{ifxetex,ifluatex}
\usepackage{fixltx2e} % provides \textsubscript
\ifnum 0\ifxetex 1\fi\ifluatex 1\fi=0 % if pdftex
  \usepackage[T1]{fontenc}
  \usepackage[utf8]{inputenc}
  \usepackage{textcomp} % provides euro and other symbols
\else % if luatex or xelatex
  \usepackage{unicode-math}
  \defaultfontfeatures{Ligatures=TeX,Scale=MatchLowercase}
\fi
% use upquote if available, for straight quotes in verbatim environments
\IfFileExists{upquote.sty}{\usepackage{upquote}}{}
% use microtype if available
\IfFileExists{microtype.sty}{%
\usepackage[]{microtype}
\UseMicrotypeSet[protrusion]{basicmath} % disable protrusion for tt fonts
}{}
\IfFileExists{parskip.sty}{%
\usepackage{parskip}
}{% else
\setlength{\parindent}{0pt}
\setlength{\parskip}{6pt plus 2pt minus 1pt}
}
\usepackage{hyperref}
\hypersetup{
            pdftitle={Introduction to reproducible research},
            pdfauthor={Noushin Nabavi \& Monica Granados},
            pdfborder={0 0 0},
            breaklinks=true}
\urlstyle{same}  % don't use monospace font for urls
\usepackage{longtable,booktabs}
% Fix footnotes in tables (requires footnote package)
\IfFileExists{footnote.sty}{\usepackage{footnote}\makesavenoteenv{longtable}}{}
\usepackage{graphicx,grffile}
\makeatletter
\def\maxwidth{\ifdim\Gin@nat@width>\linewidth\linewidth\else\Gin@nat@width\fi}
\def\maxheight{\ifdim\Gin@nat@height>\textheight\textheight\else\Gin@nat@height\fi}
\makeatother
% Scale images if necessary, so that they will not overflow the page
% margins by default, and it is still possible to overwrite the defaults
% using explicit options in \includegraphics[width, height, ...]{}
\setkeys{Gin}{width=\maxwidth,height=\maxheight,keepaspectratio}
\setlength{\emergencystretch}{3em}  % prevent overfull lines
\providecommand{\tightlist}{%
  \setlength{\itemsep}{0pt}\setlength{\parskip}{0pt}}
\setcounter{secnumdepth}{5}
% Redefines (sub)paragraphs to behave more like sections
\ifx\paragraph\undefined\else
\let\oldparagraph\paragraph
\renewcommand{\paragraph}[1]{\oldparagraph{#1}\mbox{}}
\fi
\ifx\subparagraph\undefined\else
\let\oldsubparagraph\subparagraph
\renewcommand{\subparagraph}[1]{\oldsubparagraph{#1}\mbox{}}
\fi

% set default figure placement to htbp
\makeatletter
\def\fps@figure{htbp}
\makeatother

\usepackage{booktabs}
\usepackage[]{natbib}
\bibliographystyle{plainnat}

\title{Introduction to reproducible research}
\author{Noushin Nabavi \& Monica Granados}
\date{2020-08-23}

\begin{document}
\maketitle

{
\setcounter{tocdepth}{1}
\tableofcontents
}
\hypertarget{preface}{%
\chapter*{Preface}\label{preface}}
\addcontentsline{toc}{chapter}{Preface}

A repository to house materials for a reproducibility in research workshop using \href{https://www.r-project.org/}{\texttt{R}}.

The goal of this workshop is to teach concepts of \texttt{reproducible\ research} to new-to-programming professionals.

\texttt{Reproducible\ research} is the idea that data analyses, and more generally, scientific claims, are published with their data and software code so that others may verify the findings and build upon them.

The need for reproducibility is increasing dramatically as data analyses become more complex, involving larger datasets and more sophisticated computations.

Reproducibility allows for people to focus on the actual content of a data analysis, rather than on superficial details reported in a written summary. In addition, reproducibility makes an analysis more useful to others because the data and code that actually conducted the analysis are available.

This workshop focuses on the concepts and tools behind reporting modern data analyses in a reproducible manner. As part of this, we introduce tools that enable publishing data analyses in a single document that allows others to easily execute the same analysis to obtain the same results.

Additionally, as part of this workshop, we briefly introduce data structures, importing and wrangling data, exploring missingness, as well as data visualizations and reporting.

\texttt{R} is a popular statistical computing language, commonly used in many scientific disciplines for statistical analysis, generating production-quality graphics, and automating data workflow tasks. The workshop content will follow best practices for using \texttt{R}, giving attendees a foundation in the fundamentals of R and scientific computing.

This work is licensed under the Creative Commons Attribution 4.0 International License.
To view a copy of this license, visit \url{http://creativecommons.org/licenses/by/4.0/}.

\hypertarget{introduction-to-reproducible-research}{%
\chapter{Introduction to reproducible research}\label{introduction-to-reproducible-research}}

\begin{itemize}
\tightlist
\item
  what is reproducible research?
\item
  components of reproducible research
\end{itemize}

\hypertarget{motivation}{%
\section{Motivation}\label{motivation}}

\hypertarget{tools-for-reproducible-projects}{%
\chapter{Tools for reproducible projects}\label{tools-for-reproducible-projects}}

\begin{itemize}
\item
  tools to set up a reproducible project
\item
  the required tools: e.g.~R, Rstudio, python, open-refine, good tables
\item
  notebooks (Rmarkdown, Jupyter)
\item
  demo
\end{itemize}

\hypertarget{reproducible-research-projects}{%
\chapter{Reproducible research projects}\label{reproducible-research-projects}}

\begin{itemize}
\item
  Tidy data principles
\item
  tidyr and dplyr
\item
  what are data structures?
\item
  what are data frames?
\item
  tidy principles
\item
  missingness
\item
  demo with data
\end{itemize}

\hypertarget{introduction-to-markdown}{%
\chapter{Introduction to markdown}\label{introduction-to-markdown}}

\begin{itemize}
\tightlist
\item
  how does rmarkdown work? Advantages, disadvantages
\item
  rmarkdown syntax (examples)
\end{itemize}

\hypertarget{rmarkdown-and-usage}{%
\chapter{Rmarkdown and usage}\label{rmarkdown-and-usage}}

\begin{itemize}
\tightlist
\item
  reproducible reporting with rmarkdown
\end{itemize}

\hypertarget{markdown-capabilities}{%
\chapter{markdown capabilities}\label{markdown-capabilities}}

\begin{itemize}
\tightlist
\item
  plots and graphs
\item
  reproducible code to visualize data
\item
  ggplot library in R
\item
  demo with data
\end{itemize}

\hypertarget{demo}{%
\chapter{demo}\label{demo}}

\begin{itemize}
\tightlist
\item
  demo one example
\end{itemize}

\hypertarget{reproducible-research-for-github}{%
\chapter{Reproducible research for github}\label{reproducible-research-for-github}}

\hypertarget{bibliography}{%
\chapter{Bibliography}\label{bibliography}}

\hypertarget{resources}{%
\chapter{Resources}\label{resources}}

\begin{itemize}
\tightlist
\item
  Reproducible research with R and RStudio: \url{http://christophergandrud.github.io/RepResR-RStudio/}
\item
  Tools for reproducible research: \url{https://kbroman.org/Tools4RR/}
\item
  Data privacy and security: \url{https://dataprivacymanager.net/security-vs-privacy/}
\item
  BC-Gov framework for github \url{https://github.com/bcgov/BC-Policy-Framework-For-GitHub}
\item
  Making slides with Xaringan package in RMarkdown: \url{https://arm.rbind.io/slides/xaringan.html}
\item
  Data wrangling with R: \url{https://cengel.github.io/R-data-wrangling/}
\item
  Data cleaning with R and tidyverse: \url{https://towardsdatascience.com/data-cleaning-with-r-and-the-tidyverse-detecting-missing-values-ea23c519bc62}
\item
  Gallery of missing data visualization: \url{https://cran.r-project.org/web/packages/naniar/vignettes/naniar-visualisation.html}
\item
  How does R handle missing values: \url{https://stats.idre.ucla.edu/r/faq/how-does-r-handle-missing-values/}
\end{itemize}

\end{document}
